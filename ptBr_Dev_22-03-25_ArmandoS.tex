\documentclass[a4paper,10pt]{article}
\usepackage[margin=0.5in,nofoot]{geometry}
\usepackage{fontawesome5}
\usepackage{hyperref}
\usepackage{titlesec}
\usepackage{xcolor}
\usepackage{bookmark}


\hypersetup{
    colorlinks=true,
    linkcolor=black,
    filecolor=black,
    urlcolor=black,
    citecolor=black
}

\titleformat{\section}{\large\bfseries}{\thesection}{1em}{}[\titlerule]
\titlespacing*{\section}{0pt}{*1}{*1}

\newcommand{\entry}[4]{
  \noindent\textbf{#1} \hfill #2 \\
  \noindent\textit{#3} \hfill \textit{#4} \\
  \vspace{2pt}
}

\newcommand{\project}[2]{
  \noindent\textbf{#1} \hfill #2 \\
  \vspace{2pt}
}

\begin{document}

\pagenumbering{gobble}

\noindent
\begin{minipage}[t]{0.5\textwidth}
\textbf{\Large Armando A. von Grap Solheiro}

\vspace{0.4em}

\end{minipage}%
\begin{minipage}[t]{0.5\textwidth}
\raggedleft

São Paulo - SP

{\color{blue}} \href{tel:+5591980243595}{\faPhone \space (91) 98024-3595}
{\color{blue}} \href{mailto:avgsolheiro@gmail.com}{\faEnvelope \space avgsolheiro@gmail.com}

\vspace{0.2em}
 \quad
{\color{blue}} \href{https://github.com/asolheiro}{\faGithub \space GitHub} \quad
{\color{blue}} \href{https://www.linkedin.com/in/armandosolheiro/}{\faLinkedin \space LinkedIn} \\
\end{minipage}

\vspace{1em}

\begin{center}
    \textbf{\Large Desenvolvedor de Software}
\end{center}
\vspace{0.5em}

\section*{Resumo Profissional}

Profissional de TI com foco em desenvolvimento de back-end de aplicações, com Go e Python, e no suporte a plataformas utilizando ferramentas DevOps e GitOps, como Docker, Kubernetes, Helm,  GitHub Actions e GitLabCI.

Possui sólido histórico com aplicações Web, RESTful APIs automações, integrações e sistemas de mensagerias, mas também possui experiências pontuais com front-end, sistemas embarcados e outras linguagens de programação



\vspace{0.6em}

\vspace{0.5em}

\section*{Experiência}

\vspace{0.6em}

\entry{JSRF Consulting}{\faCalendar \space Janeiro/2023 -- Dezembro/2024}{Desenvolvedor}
\space
\vspace{-1.6em}
\begin{itemize}
\setlength\itemsep{0em}
\item Desenvolvimento de ferramentas para credenciamento em eventos, reduzindo o trabalho manual e tempo do processo.
\item Desenvolvimento de SaaS para simplificar o trabalho manual em escritórios de advocacia.
\item Criação de métricas e indicadores para monitoramento de utilização do acervo.
\item Desenvolvimento de chatbots para integrar serviços de clientes.
\end{itemize}

\entry{Banco do estado do Pará}{\faCalendar \space Agosto/2024 -- Dezembro/2024}{Analista de sistemas}
\space
\vspace{-1.6em}
\begin{itemize}
\setlength\itemsep{0em}
\item Criação e padronização da documentação dos sistemas legados requisitada pelo sistema de certificação do BACEN.
\item Desenvolvimento do sistema interno de gestão de contratos de fornecedores.
\item Padronização e atualização das esteiras CI/CD de sistemas;
\end{itemize}

\entry{Jack Experts}{\faCalendar \space Dezembro/2024 -- Presente}{Analista DevOps}
\space
\vspace{-1.6em}
\begin{itemize}
\setlength\itemsep{0em}
\item Provisionamento de clusters com a stack padrão da empresa: UI para gerenciamento, monitoramento e coleta de métricas, alertas personalizados, centralização de logs, backup programado, atualização e automatização de rotinas.
\item Desenvolvimento de ferramenta para automatizar rotinas e relatórios de healthcheck.
\item Migração de clusters entre provedores na nuvem, visando o menor custo para o cliente.
\item Criação de aplicações de Cloud Native; disponibilizando,assim, pacotes compatíveis com todas as ferramentas mais utilizadas no mercado
\item Desenvolvimento de esteiras para automatizar construção, disponibilização e implementação das aplicações; minimizando, assim, o trabalho manual no processo e, portanto, os erros humanos.
\end{itemize}

% \vspace{-1.6em}

\section*{Educação}
\vspace{0.6em}

\entry{Estácio}{\faCalendar \space 2023 - 2026}{Tecnólogo em Análise e Desenvolvimento de Sistemas}{\faMapMarker \space São Paulo - SP}

\entry{Universidade da Amazônia}{\faCalendar \space 2020 - 2022}{Gastronomia}{\faMapMarker \space Belém - PA}


\section*{Habilidades e Competências}
\vspace{0.6em}
\begin{itemize}
\setlength\itemsep{0em}
\item Linguagens e Ferramentas: Python, Go, FastAPI, Django.
\item Banco de Dados: SQL, PostgreSQL, MySQL, Redis.
\item DevOps: Docker, Kubernetes, Helm, GitHub Action, GitLabCI
\item Outros: Git, GitHub, GitLab, Linux, RabbitMQ, Markdown

\end{itemize}


\section*{Idiomas}
\vspace{0.6em}
\begin{itemize}
\setlength\itemsep{0em}
\item Português: Nativo
\item Inglês: Avançado.
\item Francês: Intermediário.
\item Espanhol: Básico.



\end{itemize}

\end{document}
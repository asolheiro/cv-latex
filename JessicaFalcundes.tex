\documentclass[a4paper,10pt]{article}
\usepackage[margin=0.5in,nofoot]{geometry}
\usepackage{fontawesome5}
\usepackage{hyperref}
\usepackage{titlesec}
\usepackage{xcolor}

\hypersetup{
    colorlinks=true,
    linkcolor=black,
    filecolor=black,
    urlcolor=black,
    citecolor=black
}

\titleformat{\section}{\large\bfseries}{\thesection}{1em}{}[\titlerule]
\titlespacing*{\section}{0pt}{*1}{*1}

\newcommand{\entry}[4]{
  \noindent\textbf{#1} \hfill #2 \\
  \noindent\textit{#3} \hfill \textit{#4} \\
  \vspace{2pt}
}

\newcommand{\project}[2]{
  \noindent\textbf{#1} \hfill #2 \\
  \vspace{2pt}
}

\begin{document}

\pagenumbering{gobble}

\noindent
\begin{minipage}[t]{0.5\textwidth}
\textbf{\Large Jessica Aline Barros Falcundes}

\vspace{0.4em}

\end{minipage}%
\begin{minipage}[t]{0.5\textwidth}
\raggedleft

São Paulo - SP

{\color{blue}} \href{tel:+5567998748431}{\faPhone \space (11) 99767-0355}
{\color{blue}} \href{mailto:jessicafalcundes@outlook.com}{\faEnvelope \space jessicafalcundes@outlook.com}

\vspace{0.2em}
 \quad
{\color{blue}} \href{https://github.com/jessicafalcundes}{\faGithub \space GitHub} \quad
{\color{blue}} \href{https://www.linkedin.com/in/jessicafalcundes/}{\faLinkedin \space LinkedIn} \\
\end{minipage}

\vspace{1em}

\begin{center}
    \textbf{\Large Analista de Dados Jr.}
\end{left}
\vspace{0.5em}

\section*{Resumo Profissional}

\vspace{0.6em}

Graduanda em Análise e Desenvolvimento de Sistemas pela Fatec Carapicuíba, com foco em análise de dados. Graduada em História pela Universidade Santo Amaro - UNISA. Atuou como pesquisadora no Grupo de Pesquisa Ciência, Saúde, Gênero e Sentimento - CISGES/UNISA/CNPq, contribuindo para o projeto "História dos Negros na Literatura Brasileira: representações e formas de subjetividade". Possui experiência em pesquisa e análise de dados, com ênfase na organização e interpretação de informações complexas. 

\vspace{0.6em}

\vspace{0.5em}

\section*{Experiência}

\vspace{0.6em}

\entry{Colégio Santa Cruz }{\faCalendar \space Setembro/2021 -- Setembro/2024}{Arquivista}
\space
\vspace{-1.6em}
\begin{itemize}
\setlength\itemsep{0em}
\item Gerenciamento de acervos, garantindo organização e acessibilidade para projetos educacionais.
\item Elaboração de relatórios detalhados sobre movimentação de documentos e recursos.
\item Criação de métricas e indicadores para monitoramento de utilização do acervo.
\item Atuação em processos de desbaste e preservação de obras raras.

\end{itemize}

\entry{Universidade Santo Amaro}{\faCalendar \space 
 Janeiro/2020 -- Setembro/2021}{Auxiliar de biblioteca}
\space
\vspace{-1.6em}
\begin{itemize}
\setlength\itemsep{0em}
\item Suporte no atendimento aos usuários e gestão de documentação.
\item Colaboração em processos de aquisição, classificação e conservação de materiais.
\item Implementação de melhorias no fluxo de informação e acesso aos recursos da biblioteca.

\end{itemize}

\entry{Grupo de Pesquisa CISGES/UNISA/CNPq}{\faCalendar \space 
 Julho/2017 -- Dezembro/2018}{Pesquisadora}{Projeto: "História dos Negros na Literatura Brasileira: Representações e Formas de Subjetividade"}
\space
\vspace{-1.6em}
\begin{itemize}
\setlength\itemsep{0em}
\item Levantamento, categorizacão e análise de dados qualitativos e quantitativos relacionados a representações sociais e históricas.
\item Revisão bibliográfica e análise documental de publicações temáticas.
\item Produção de relatórios técnico-científicos e artigos para disseminação do conhecimento.
\item Utilização de ferramentas colaborativas para organização e documentação da pesquisa.

\end{itemize}



% \vspace{-1.6em}

\section*{Educação}
\vspace{0.6em}

\entry{FATEC Carapicuíba}{\faCalendar \space 2023 - 2026}{Tecnólogo em Análise e Desenvolvimento de Sistemas}{\faMapMarker \space São Paulo - SP}

\entry{Universidade Santo Amaro}{\faCalendar \space 2016 - 2018}{Licenciatura em História}{\faMapMarker \space São Paulo - SP}


\section*{Habilidades e Competências}
\vspace{0.6em}
\begin{itemize}
\setlength\itemsep{0em}
\item Linguagens e Ferramentas: Python, SQL, Power BI, Excel Avançado, VSCode.
\item Banco de Dados: PostgreSQL, BigQuery, NoSQL.
\item Computação em Nuvem: Azure, AWS, Google Cloud.
\item Controle de Versão: Git, GitHub.
\item Metodologias ágeis: Scrum.
\item Análise de dados e Pesquisa: análise quali-quantitativa de dados, categorizacão e interpretação de informações, elaboração de relatórios técnicos e científicos, revisão bibliográfica e análise documental, domínio de ferramentas colaborativas para pesquisa, organização e gestão de informações.

\end{itemize}


\section*{Idiomas}
\vspace{0.6em}
\begin{itemize}
\setlength\itemsep{0em}
\item Inglês: Avançado.
\item Espanhol: Intermediário.



\end{itemize}

\end{document}